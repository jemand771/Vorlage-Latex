
% hilfsbefehle für singular/plural
\newcommand{\fmtn}{\pyc{print("n" if len(autoren) > 1 else "{}", end="")}}
\newcommand{\fmtsub}{\pyc{print("wir" if len(autoren) > 1 else "ich", end="")}}
\newcommand{\fmtSub}{\pyc{print("Wir" if len(autoren) > 1 else "Ich", end="")}}
\newcommand{\fmtmy}{\pyc{print("unsere Belegarbeit" if len(autoren) > 1 else "meinen Praxisbeleg", end="")}}

%	Eidesstattliche Erklärung

\cleardoublepage 
\section{Ehrenwörtliche Erklärung}
\vspace*{1cm}
\begin{center}
\huge\textbf{Ehrenwörtliche Erklärung}\\
\end{center}
\vspace*{1cm}
\normalsize
\fmtSub erkläre\fmtn hiermit ehrenwörtlich,

\begin{enumerate}
	\vspace{1cm}
	\item dass \fmtsub \fmtmy mit dem Thema:\\
	
	\textbf{\pyc{print(data["titel"], end="")}}\\

	ohne fremde Hilfe angefertigt habe\pyc{print("n" if len(autoren) > 1 else "", end=",")}
	\item dass \fmtsub die Übernahme wörtlicher Zitate aus der Literatur sowie die\\ 		  
	Verwendung der Gedanken anderer Autoren an den entsprechenden\\
	Stellen innerhalb der Arbeit gekennzeichnet habe\fmtn und
	\item dass \fmtsub \fmtmy bei keiner anderen Prüfung vorgelegt habe\pyc{print("n" if len(autoren) > 1 else "", end=".")}\\[1,5cm]
\end{enumerate}
\fmtSub \pyc{print("sind uns" if len(autoren) > 1 else "bin mir", end="")}bewusst, dass eine falsche Erklärung rechtliche Folgen haben wird.\\[1,5cm]
		
\vfill

Glauchau, \abgabedatum\newline\noindent\rule{0.35\columnwidth}{0.4pt}\hspace{0.05\columnwidth}\rule{0.6\columnwidth}{0.4pt}\\
Ort, Datum\hspace{0.27\columnwidth}Unterschrift\pyc{if len(autoren) > 1: print("en", end="")}

{\footnotesize Dies ist eine zur Nutzung mit \pyc{print("mehreren Autoren und " if len(autoren) > 1 else "", end="")}\LaTeX\ angepasste Version der in \href{https://www.ba-glauchau.de/fileadmin/glauchau/waehrend-des-studium/dokumente/pruefungen/4BA-F.207_Hinweise_zur_Anfertigung_wissenschaftlicher_Arbeiten.pdf}{Anhang 4 der Hinweise zur Anfertigung wissenschaftlicher Arbeiten an der Staatlichen Studienakademie Glauchau vorgegebenen ehrenwörtlichen Erklärung.}}

