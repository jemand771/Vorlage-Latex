%Titelseite

\begin{titlepage}
\begin{pycode}
import json
import os
import sys

pytex.add_dependencies("meta.json")

ST_KEYS = (
        ("studiengang", "studienrichtung", "seminargruppe"),  # schlüssel
        ("Studiengang", "Studienrichtung", "Seminargruppe"),  # einzahl
        ("Studiengänge", "Studienrichtungen", "Seminargruppen")  # mehrzahl
    )

VSPACE = 0.7

blockify = lambda lst, text: [("" if i else text, x) for i, x in enumerate(lst)]


with open("meta.json") as f:
    data = json.load(f)
# TODO default values (error) everywhere
data.setdefault("is_beleg", False)
beleg = data["is_beleg"]
data.setdefault("datum", "??datum??")
data.setdefault("titel", "??titel??")
data.setdefault("autoren", [{}])
autoren = data["autoren"]
for i, x in enumerate(autoren):
    for field in ("name", "matnum", *ST_KEYS[0]):
        x.setdefault(field, "??autor" + str(i) + "-" + field + "??")
autoren = sorted(autoren, key=lambda x: x["name"].split(" ")[-1])

if beleg:
    data.setdefault("beleg_info", {})
    data["beleg_info"].setdefault("betreuer", "??betreuer??")
    data["beleg_info"].setdefault("anschrift", ["??anschrift1??", "??anschrift2??"])
    data["beleg_info"]["firma"].setdefault("name", "??firma??")
    data["beleg_info"]["firma"].setdefault("anschrift", ["??firmen-anschrift1??", "??firmen-anschrift2??"])
    autoren = autoren[:1]

    # TODO convert date to local if regex match
def make_titelseite():    
    # überschrift und titel
    print("\\begin{center}")
    print("\\textbf{\\Huge\\\\ " + (
        "Praxisbeleg" if beleg else "Projektarbeit") + "\\\\}")
    print("\\vspace{1.5cm}")
    print("\\LARGE{" + data.get("titel", "404: titel") + " \\\\}")
    print("\\vspace{1.5cm}")
    print("\\end{center}")
    print("\\begin{flushleft}")
    print("\\large{\\begin{tabular}{l l r}")
    #print("\\vspace{1.0cm}")
    # TODO spacing einheitlich machen (praxisbeleg sieht schöner aus,projektarbeit     	braucht aber evtl. mehr platz)
    # TODO ab hier (tabelle) weiter machen

    # bei projektarbeit alle autoren auflisten, beim praxisbeleg den autor mit anschrift
    autoren_block = list(map(lambda x: bold(x["name"]), autoren))
    if beleg:
        autoren_block.extend(data["beleg_info"]["anschrift"])
    autoren_block = blockify(autoren_block, "Von")

    st_data = [sorted(set(map(lambda x: x[key], autoren))) for key in ST_KEYS[0]]

    studiengang_dinge = [(
        ST_KEYS[1 + (len(x) > 1)][i],
        ", ".join(x)
    ) for i, x in enumerate(st_data)]

    matnum_block = blockify(list(map(str, map(lambda x: x["matnum"], autoren))), "Matrikelnummer" + ("n" if len(autoren) > 1 else ""))

    pp_block = [VSPACE]
    gutachter = [ga[key] for key in ("name", "institution") for ga in data["gutachter"]]
    if beleg:
        pp = data["beleg_info"]["firma"]
        pp_block.extend(blockify([pp["name"]] + pp["anschrift"], "Praxispartner"))
        pp_block.append(VSPACE)
        gutachter = [data["beleg_info"]["betreuer"], pp["name"]] + gutachter

    gutachter_block = blockify(gutachter, "Gutachter")

    print(make_table([
        ("Vorgelegt am", data["datum"]),
        VSPACE,
        *autoren_block,
        VSPACE,
        *studiengang_dinge,
        VSPACE,
        *matnum_block,
        *pp_block,  # ist entweder VSPACE or VSPACE, praxispartner, VSPACE
        *gutachter_block
        #("Studiengang", data["studiengang"]),
        #("Studienrichtung", data["studienrichtung"]),
        #("Seminargruppe", ", ".join(sem_gruppen))
    ]))

    #\textbf{Vorgelegt am:}  ~ & \abgabedatum\\
#
#\textbf{Von:}           ~ & \textbf{\autoreins}\\
#                        ~ & \textbf{\autorzwei}\\
#\vspace{1.0cm}
#                        ~ & \textbf{\autordrei}\\
#
#\textbf{Studiengang:}   ~ & \studiengang \\
#\vspace{1.0cm}
#\textbf{Studienrichtung:} ~ & \studienrichtung \\
#\vspace{1.0cm}
#\textbf{Seminargruppe:} ~ & \seminargruppe \\
#
#\textbf{Matrikelnummer:} ~ & \matnumeins \\
#                         ~ & \matnumzwei \\
#\vspace{1.0cm}
#                         ~ & \matnumdrei \\
#\textbf{Gutachter:}     ~ & \betreuereins \\ ~ & (\institutioneins)\\
#                        ~ & \betreuerzwei \\ ~ & (\institutionzwei)\\

    print("\\end{tabular}}")  # end of \large{
    print("\\end{flushleft}")

def bold(text):
    return "\\textbf{" + text + "}"
    	
def make_tableline(key, value):
    return ("\\textbf{" + key + ":}" if key else "") + " ~ & " + value + "\\\\"

def make_vspace(space):
    return "\\vspace{" + str(float(space)) + "cm}"

# NOTE putting multiple spaces (numbers) back to back causes undefined behavior
def make_table(lst):
    def is_numtype(num):
        return isinstance(num, float) or isinstance(num, int)
    lines = []
    for i, entry in enumerate(lst):
        if is_numtype(entry) or entry is None:
            continue
        if i != len(lst) - 1:
            if is_numtype(lst[i+1]):
                lines.append(make_vspace(lst[i+1]))
        lines.append(make_tableline(*entry))
    return "\n".join(lines)

make_titelseite()

\end{pycode}
\end{titlepage}
\newpage